\documentclass[11pt]{report}
\usepackage[T2A]{fontenc}
\usepackage[utf8]{inputenc}
\usepackage[russian]{babel}
\usepackage{amsmath,amssymb}
\usepackage{graphicx}
\usepackage{float}
\oddsidemargin=-19mm
\topmargin=-30mm
\textheight 26cm
\hsize 18cm
\textwidth 20cm
\begin{document}

\pagestyle{empty}
{\bf Индивидуальное задание.}

Исследовать функцию f(x) с помощью производной, найти необходимые пределы и решить уравнения. Построить график функции и асимптот (если есть), отметить и подписать точки экстремума и точки перегиба (если есть), включить функцию и асимптоты (если есть) в легенду.

\begin{center}
\noindent\rule{8cm}{0.4pt}
\end{center}
Вариант $0$


$$f(x) = \frac{\left(2 x + 3\right) \operatorname{atan}{\left(x \right)} - 1}{2 x + 3}$$
\begin{center}
\noindent\rule{8cm}{0.4pt}
\end{center}
Вариант $1$


$$f(x) = \frac{4 x^{3} + 18 x^{2} - 168 x + 63}{6 \cdot \left(2 x - 9\right)}$$
\begin{center}
\noindent\rule{8cm}{0.4pt}
\end{center}
Вариант $2$


$$f(x) = - \frac{2 e^{x}}{3 x - 3}$$
\begin{center}
\noindent\rule{8cm}{0.4pt}
\end{center}
Вариант $3$


$$f(x) = \frac{\left(2 - x\right) \left(x - 1\right) + \frac{2}{3}}{x - 1}$$
\begin{center}
\noindent\rule{8cm}{0.4pt}
\end{center}
Вариант $4$


$$f(x) = - \frac{2}{x - 6}$$
\begin{center}
\noindent\rule{8cm}{0.4pt}
\end{center}
Вариант $5$


$$f(x) = \frac{- 2 \cdot \left(3 x + 1\right) \operatorname{atan}{\left(x \right)} + 3}{3 x + 1}$$
\begin{center}
\noindent\rule{8cm}{0.4pt}
\end{center}
Вариант $6$


$$f(x) = - \frac{12 e^{x}}{4 x - 9}$$
\begin{center}
\noindent\rule{8cm}{0.4pt}
\end{center}
Вариант $7$


$$f(x) = \frac{2 \left(\left(4 x - 3\right) \operatorname{atan}{\left(\frac{x}{2} \right)} - 3\right)}{4 x - 3}$$
\begin{center}
\noindent\rule{8cm}{0.4pt}
\end{center}
Вариант $8$


$$f(x) = \frac{3 x - 2}{9 \left(x + 2\right)}$$
\begin{center}
\noindent\rule{8cm}{0.4pt}
\end{center}
Вариант $9$


$$f(x) = \frac{9 \left(- 2 x^{2} - 1\right)}{4 \left(x - 1\right)}$$
\begin{center}
\noindent\rule{8cm}{0.4pt}
\end{center}
Вариант $10$


$$f(x) = - \frac{4 e^{- x}}{3 \cdot \left(4 x - 1\right)}$$
\begin{center}
\noindent\rule{8cm}{0.4pt}
\end{center}
Вариант $11$


$$f(x) = - \frac{e^{x}}{2 x + 2}$$
\begin{center}
\noindent\rule{8cm}{0.4pt}
\end{center}
Вариант $12$


$$f(x) = - \frac{3 e^{x}}{x - 3}$$
\begin{center}
\noindent\rule{8cm}{0.4pt}
\end{center}
Вариант $13$


$$f(x) = - \frac{2 e^{x}}{3 x - 1}$$
\begin{center}
\noindent\rule{8cm}{0.4pt}
\end{center}
Вариант $14$


$$f(x) = \frac{3 e^{x}}{2 \cdot \left(3 x - 1\right)}$$
\begin{center}
\noindent\rule{8cm}{0.4pt}
\end{center}
Вариант $15$


$$f(x) = - \frac{3 e^{x}}{2 x - 4}$$
\begin{center}
\noindent\rule{8cm}{0.4pt}
\end{center}
Вариант $16$


$$f(x) = - \frac{2 e^{x}}{x - 2}$$
\begin{center}
\noindent\rule{8cm}{0.4pt}
\end{center}
Вариант $17$


$$f(x) = - \frac{x^{2}}{3 x + 2}$$
\begin{center}
\noindent\rule{8cm}{0.4pt}
\end{center}
Вариант $18$


$$f(x) = \frac{- x^{2} - \frac{3 x}{2} + 3}{x - 2}$$
\begin{center}
\noindent\rule{8cm}{0.4pt}
\end{center}
Вариант $19$


$$f(x) = \frac{3}{2 \cdot \left(6 x - 1\right)}$$
\begin{center}
\noindent\rule{8cm}{0.4pt}
\end{center}
Вариант $20$


$$f(x) = - x + \frac{1}{x - \frac{3}{2}}$$
\begin{center}
\noindent\rule{8cm}{0.4pt}
\end{center}
Вариант $21$


$$f(x) = \frac{6 x - 7}{2 x - 3}$$
\begin{center}
\noindent\rule{8cm}{0.4pt}
\end{center}
Вариант $22$


$$f(x) = \frac{2 \left(x - 3\right) \left(x - 1\right) + 1}{x - 3}$$
\begin{center}
\noindent\rule{8cm}{0.4pt}
\end{center}
Вариант $23$


$$f(x) = \frac{4 x - 5}{4 x - 3}$$
\begin{center}
\noindent\rule{8cm}{0.4pt}
\end{center}
Вариант $24$


$$f(x) = \frac{x \left(1 - x\right)}{2 x - 1}$$
\begin{center}
\noindent\rule{8cm}{0.4pt}
\end{center}
Вариант $25$


$$f(x) = \frac{e^{- x}}{2 x - 3}$$
\begin{center}
\noindent\rule{8cm}{0.4pt}
\end{center}
Вариант $26$


$$f(x) = - \frac{e^{x}}{3 x - 6}$$
\begin{center}
\noindent\rule{8cm}{0.4pt}
\end{center}
Вариант $27$


$$f(x) = \frac{e^{- x}}{x + 1}$$
\begin{center}
\noindent\rule{8cm}{0.4pt}
\end{center}
Вариант $28$


$$f(x) = \frac{\left(2 x - 9\right) \left(x^{2} - x - 1\right) + 6}{2 x - 9}$$
\begin{center}
\noindent\rule{8cm}{0.4pt}
\end{center}
Вариант $29$


$$f(x) = - \frac{9 e^{- x}}{2 x - 3}$$
\begin{center}
\noindent\rule{8cm}{0.4pt}
\end{center}
Вариант $30$


$$f(x) = \frac{3 x^{2} - x + 1}{2 \left(x + 1\right)}$$
\begin{center}
\noindent\rule{8cm}{0.4pt}
\end{center}
Вариант $31$


$$f(x) = - \frac{3 e^{- x}}{x + 3}$$
\begin{center}
\noindent\rule{8cm}{0.4pt}
\end{center}
Вариант $32$


$$f(x) = \frac{2 \left(- 3 x^{2} + 3 x - 1\right)}{3 \cdot \left(3 x - 2\right)}$$
\begin{center}
\noindent\rule{8cm}{0.4pt}
\end{center}
Вариант $33$


$$f(x) = - \frac{6 e^{- x}}{9 x - 2}$$
\begin{center}
\noindent\rule{8cm}{0.4pt}
\end{center}
Вариант $34$


$$f(x) = \frac{9 e^{- x}}{2 x + 3}$$
\begin{center}
\noindent\rule{8cm}{0.4pt}
\end{center}
Вариант $35$


$$f(x) = \frac{2 \cdot \left(2 - 9 x\right)}{3 \left(x - 4\right)}$$
\begin{center}
\noindent\rule{8cm}{0.4pt}
\end{center}
Вариант $36$


$$f(x) = - \frac{4 e^{- x}}{2 x + 9}$$
\begin{center}
\noindent\rule{8cm}{0.4pt}
\end{center}
Вариант $37$


$$f(x) = - \frac{4 e^{- x}}{3 \cdot \left(3 x + 2\right)}$$
\begin{center}
\noindent\rule{8cm}{0.4pt}
\end{center}
Вариант $38$


$$f(x) = - \frac{2 e^{x}}{3 x - 9}$$
\begin{center}
\noindent\rule{8cm}{0.4pt}
\end{center}
Вариант $39$


$$f(x) = \frac{6 x^{2} + x - 6}{3 x + 1}$$
\begin{center}
\noindent\rule{8cm}{0.4pt}
\end{center}
Вариант $40$


$$f(x) = - \frac{4 x^{2}}{2 x + 1}$$
\begin{center}
\noindent\rule{8cm}{0.4pt}
\end{center}
Вариант $41$


$$f(x) = - \frac{9 e^{- x}}{2 \cdot \left(2 x + 9\right)}$$
\begin{center}
\noindent\rule{8cm}{0.4pt}
\end{center}
Вариант $42$


$$f(x) = \frac{x^{2} \cdot \left(3 - x\right) + 1}{x - 3}$$
\begin{center}
\noindent\rule{8cm}{0.4pt}
\end{center}
Вариант $43$


$$f(x) = \frac{2 \cdot \left(3 - x\right)}{2 x + 3}$$
\begin{center}
\noindent\rule{8cm}{0.4pt}
\end{center}
Вариант $44$


$$f(x) = - \frac{9 e^{x}}{2 x - 2}$$
\begin{center}
\noindent\rule{8cm}{0.4pt}
\end{center}
Вариант $45$


$$f(x) = \frac{6 x + 1}{2 x - 1}$$
\begin{center}
\noindent\rule{8cm}{0.4pt}
\end{center}
Вариант $46$


$$f(x) = \frac{2 x^{2} - 9 x - 18}{3 \cdot \left(3 x - 1\right)}$$
\begin{center}
\noindent\rule{8cm}{0.4pt}
\end{center}
Вариант $47$


$$f(x) = \frac{- 2 x^{2} - 2 x + 3}{3 \left(x + 2\right)}$$
\begin{center}
\noindent\rule{8cm}{0.4pt}
\end{center}
Вариант $48$


$$f(x) = - \frac{e^{x}}{4 x - 3}$$
\begin{center}
\noindent\rule{8cm}{0.4pt}
\end{center}
Вариант $49$


$$f(x) = - \frac{e^{x}}{3 x + 1}$$
\begin{center}
\noindent\rule{8cm}{0.4pt}
\end{center}
Вариант $50$


$$f(x) = \frac{9 e^{- x}}{6 x + 1}$$
\begin{center}
\noindent\rule{8cm}{0.4pt}
\end{center}
Вариант $51$


$$f(x) = \frac{6 x^{2} + 1}{9 \left(x - 1\right)}$$
\begin{center}
\noindent\rule{8cm}{0.4pt}
\end{center}
Вариант $52$


$$f(x) = \frac{3 \left(- x^{2} + x + 2\right)}{3 x - 1}$$
\begin{center}
\noindent\rule{8cm}{0.4pt}
\end{center}
Вариант $53$


$$f(x) = \frac{2 \cdot \left(3 x + 1\right) \operatorname{atan}{\left(x \right)} - 1}{3 x + 1}$$
\begin{center}
\noindent\rule{8cm}{0.4pt}
\end{center}
Вариант $54$


$$f(x) = \frac{\left(4 - 9 x\right) \left(x - 2\right) + 12}{6 \left(x - 2\right)}$$
\begin{center}
\noindent\rule{8cm}{0.4pt}
\end{center}
Вариант $55$


$$f(x) = - \frac{3 e^{- x}}{2 \left(x - 3\right)}$$
\begin{center}
\noindent\rule{8cm}{0.4pt}
\end{center}
Вариант $56$


$$f(x) = \frac{3 \cdot \left(1 - x\right) \left(x - 3\right) - 1}{x - 3}$$
\begin{center}
\noindent\rule{8cm}{0.4pt}
\end{center}
Вариант $57$


$$f(x) = \frac{6 x^{2} + 4 x - 3}{2 \left(x - 1\right)}$$
\begin{center}
\noindent\rule{8cm}{0.4pt}
\end{center}
Вариант $58$


$$f(x) = \frac{x^{2} - 2 x + 2}{2 x - 3}$$
\begin{center}
\noindent\rule{8cm}{0.4pt}
\end{center}
Вариант $59$


$$f(x) = \frac{2 \left(x^{2} + x - 1\right)}{3 \cdot \left(2 x - 1\right)}$$
\begin{center}
\noindent\rule{8cm}{0.4pt}
\end{center}
Вариант $60$


$$f(x) = \frac{- 4 x^{2} - 3 x + 6}{6 \cdot \left(2 x - 1\right)}$$
\begin{center}
\noindent\rule{8cm}{0.4pt}
\end{center}
Вариант $61$


$$f(x) = \frac{x \left(4 - x\right)}{2 \left(x + 3\right)}$$
\begin{center}
\noindent\rule{8cm}{0.4pt}
\end{center}
Вариант $62$


$$f(x) = \frac{\left(2 x + 3\right) \operatorname{atan}{\left(x \right)} - 6}{2 x + 3}$$
\begin{center}
\noindent\rule{8cm}{0.4pt}
\end{center}
Вариант $63$


$$f(x) = - \frac{3 e^{- x}}{2 \cdot \left(3 x + 2\right)}$$
\begin{center}
\noindent\rule{8cm}{0.4pt}
\end{center}
Вариант $64$


$$f(x) = \frac{- 6 x^{2} + 4 x + 9}{3 \cdot \left(3 x - 2\right)}$$
\begin{center}
\noindent\rule{8cm}{0.4pt}
\end{center}
Вариант $65$


$$f(x) = \frac{\left(x - 3\right) \left(3 x^{2} + 3 x + 2\right) + 9}{3 \left(x - 3\right)}$$
\begin{center}
\noindent\rule{8cm}{0.4pt}
\end{center}
Вариант $66$


$$f(x) = \frac{- 6 x^{2} - 4 x + 3}{9 x - 2}$$
\begin{center}
\noindent\rule{8cm}{0.4pt}
\end{center}
Вариант $67$


$$f(x) = \frac{3 e^{- x}}{2 x + 3}$$
\begin{center}
\noindent\rule{8cm}{0.4pt}
\end{center}
Вариант $68$


$$f(x) = \frac{2 \left(x - 1\right) \operatorname{atan}{\left(x \right)} + 1}{x - 1}$$
\begin{center}
\noindent\rule{8cm}{0.4pt}
\end{center}
Вариант $69$


$$f(x) = \frac{\left(9 x - 1\right) \operatorname{atan}{\left(3 x \right)} - 3}{9 x - 1}$$
\begin{center}
\noindent\rule{8cm}{0.4pt}
\end{center}
Вариант $70$


$$f(x) = - \frac{6 e^{x}}{3 x - 2}$$
\begin{center}
\noindent\rule{8cm}{0.4pt}
\end{center}
Вариант $71$


$$f(x) = - \frac{3 e^{x}}{2 x + 4}$$
\begin{center}
\noindent\rule{8cm}{0.4pt}
\end{center}
Вариант $72$


$$f(x) = \frac{2 e^{x}}{9 x + 2}$$
\begin{center}
\noindent\rule{8cm}{0.4pt}
\end{center}
Вариант $73$


$$f(x) = - \frac{3 e^{- x}}{2 \left(x + 3\right)}$$
\begin{center}
\noindent\rule{8cm}{0.4pt}
\end{center}
Вариант $74$


$$f(x) = \frac{2 x \left(3 x - 2\right)}{3 \left(x + 2\right)}$$
\begin{center}
\noindent\rule{8cm}{0.4pt}
\end{center}
Вариант $75$


$$f(x) = - \frac{3 e^{x}}{12 x + 4}$$
\begin{center}
\noindent\rule{8cm}{0.4pt}
\end{center}
Вариант $76$


$$f(x) = \frac{- \left(x + 1\right) \operatorname{atan}{\left(3 x \right)} + 1}{x + 1}$$
\begin{center}
\noindent\rule{8cm}{0.4pt}
\end{center}
Вариант $77$


$$f(x) = \frac{3 \left(- 3 x^{2} + 6 x - 4\right)}{2 \cdot \left(3 x - 2\right)}$$
\begin{center}
\noindent\rule{8cm}{0.4pt}
\end{center}
Вариант $78$


$$f(x) = \frac{3 x^{2} + 4}{2 \left(x + 9\right)}$$
\begin{center}
\noindent\rule{8cm}{0.4pt}
\end{center}
Вариант $79$


$$f(x) = \frac{- 2 x^{2} - 3}{2 x - 3}$$
\begin{center}
\noindent\rule{8cm}{0.4pt}
\end{center}
Вариант $80$


$$f(x) = \frac{x + 3}{3 \left(x - 1\right)}$$
\begin{center}
\noindent\rule{8cm}{0.4pt}
\end{center}
Вариант $81$


$$f(x) = \frac{- 8 x^{2} - 2 x + 5}{2 \cdot \left(4 x - 1\right)}$$
\begin{center}
\noindent\rule{8cm}{0.4pt}
\end{center}
Вариант $82$


$$f(x) = \frac{e^{- x}}{x + 1}$$
\begin{center}
\noindent\rule{8cm}{0.4pt}
\end{center}
Вариант $83$


$$f(x) = \frac{6 e^{x}}{3 x - 2}$$
\begin{center}
\noindent\rule{8cm}{0.4pt}
\end{center}
Вариант $84$


$$f(x) = - \frac{2 e^{- x}}{3 \left(x - 1\right)}$$
\begin{center}
\noindent\rule{8cm}{0.4pt}
\end{center}
Вариант $85$


$$f(x) = \frac{e^{x}}{3 x + 2}$$
\begin{center}
\noindent\rule{8cm}{0.4pt}
\end{center}
Вариант $86$


$$f(x) = \frac{2 e^{- x}}{3 \left(x - 2\right)}$$
\begin{center}
\noindent\rule{8cm}{0.4pt}
\end{center}
Вариант $87$


$$f(x) = \frac{- 6 x^{2} + x + 8}{3 x - 2}$$
\begin{center}
\noindent\rule{8cm}{0.4pt}
\end{center}
Вариант $88$


$$f(x) = \frac{2 e^{- x}}{3 \cdot \left(2 x - 1\right)}$$
\begin{center}
\noindent\rule{8cm}{0.4pt}
\end{center}
Вариант $89$


$$f(x) = \frac{x \left(3 x + 4\right)}{2 \left(x + 3\right)}$$
\begin{center}
\noindent\rule{8cm}{0.4pt}
\end{center}
Вариант $90$


$$f(x) = \frac{e^{x}}{x + 1}$$
\begin{center}
\noindent\rule{8cm}{0.4pt}
\end{center}
Вариант $91$


$$f(x) = - \frac{e^{- x}}{x + 1}$$
\begin{center}
\noindent\rule{8cm}{0.4pt}
\end{center}
Вариант $92$


$$f(x) = \frac{3 e^{- x}}{3 x + 2}$$
\begin{center}
\noindent\rule{8cm}{0.4pt}
\end{center}
Вариант $93$


$$f(x) = \frac{3 \left(- 2 x^{2} - 2 x - 1\right)}{2 x + 1}$$
\begin{center}
\noindent\rule{8cm}{0.4pt}
\end{center}
Вариант $94$


$$f(x) = \frac{\left(x + 1\right) \operatorname{atan}{\left(x \right)} - 1}{x + 1}$$
\begin{center}
\noindent\rule{8cm}{0.4pt}
\end{center}
Вариант $95$


$$f(x) = \frac{3 x^{2} + 2 x + 4}{6 \left(x - 1\right)}$$
\begin{center}
\noindent\rule{8cm}{0.4pt}
\end{center}
Вариант $96$


$$f(x) = \frac{2 \left(x^{2} + 3 x + 1\right)}{3 \left(x + 1\right)}$$
\begin{center}
\noindent\rule{8cm}{0.4pt}
\end{center}
Вариант $97$


$$f(x) = \frac{- \left(x - 1\right) \operatorname{atan}{\left(x \right)} + 3}{x - 1}$$
\begin{center}
\noindent\rule{8cm}{0.4pt}
\end{center}
Вариант $98$


$$f(x) = \frac{4 x - 15}{4 x - 9}$$
\begin{center}
\noindent\rule{8cm}{0.4pt}
\end{center}
Вариант $99$


$$f(x) = \frac{2 e^{- x}}{3 \left(x - 1\right)}$$
\begin{center}
\noindent\rule{8cm}{0.4pt}
\end{center}
Вариант $100$


$$f(x) = - \frac{2 e^{x}}{4 x - 1}$$
\begin{center}
\noindent\rule{8cm}{0.4pt}
\end{center}
Вариант $101$


$$f(x) = \frac{3 \left(- 3 x^{2} + 2 x + 3\right)}{4 \left(x + 3\right)}$$
\begin{center}
\noindent\rule{8cm}{0.4pt}
\end{center}
Вариант $102$


$$f(x) = - \frac{9 e^{x}}{2 x - 3}$$
\begin{center}
\noindent\rule{8cm}{0.4pt}
\end{center}
Вариант $103$


$$f(x) = \frac{6 e^{- x}}{3 x - 2}$$
\begin{center}
\noindent\rule{8cm}{0.4pt}
\end{center}
Вариант $104$


$$f(x) = \frac{e^{- x}}{2 x + 3}$$
\begin{center}
\noindent\rule{8cm}{0.4pt}
\end{center}
Вариант $105$


$$f(x) = \frac{3 e^{x}}{2 \left(x - 1\right)}$$
\begin{center}
\noindent\rule{8cm}{0.4pt}
\end{center}
Вариант $106$


$$f(x) = \frac{- 3 x^{2} + 2 x + 4}{3 x + 1}$$
\begin{center}
\noindent\rule{8cm}{0.4pt}
\end{center}
Вариант $107$


$$f(x) = \frac{e^{x}}{x + 2}$$
\begin{center}
\noindent\rule{8cm}{0.4pt}
\end{center}
Вариант $108$


$$f(x) = \frac{12 - x^{2}}{x - 3}$$
\begin{center}
\noindent\rule{8cm}{0.4pt}
\end{center}
Вариант $109$


$$f(x) = \frac{27 - 2 x^{2}}{3 \left(x - 3\right)}$$
\begin{center}
\noindent\rule{8cm}{0.4pt}
\end{center}
Вариант $110$


$$f(x) = - \frac{6 e^{x}}{3 x - 2}$$
\begin{center}
\noindent\rule{8cm}{0.4pt}
\end{center}
Вариант $111$


$$f(x) = \frac{x^{2} + \frac{x}{3} - 1}{x + 3}$$
\begin{center}
\noindent\rule{8cm}{0.4pt}
\end{center}
Вариант $112$


$$f(x) = - \frac{6 e^{- x}}{x + 3}$$
\begin{center}
\noindent\rule{8cm}{0.4pt}
\end{center}
Вариант $113$


$$f(x) = \frac{2 e^{x}}{x - 2}$$
\begin{center}
\noindent\rule{8cm}{0.4pt}
\end{center}
Вариант $114$


$$f(x) = \frac{x \left(- 6 x - 13\right)}{3 x + 2}$$
\begin{center}
\noindent\rule{8cm}{0.4pt}
\end{center}
Вариант $115$


$$f(x) = \frac{2 \cdot \left(4 - 3 x\right)}{3 \cdot \left(6 x + 1\right)}$$
\begin{center}
\noindent\rule{8cm}{0.4pt}
\end{center}
Вариант $116$


$$f(x) = \frac{e^{- x}}{3 \left(x - 2\right)}$$
\begin{center}
\noindent\rule{8cm}{0.4pt}
\end{center}
Вариант $117$


$$f(x) = \frac{- \left(x + 1\right) \operatorname{atan}{\left(x \right)} + 1}{x + 1}$$
\begin{center}
\noindent\rule{8cm}{0.4pt}
\end{center}
Вариант $118$


$$f(x) = \frac{2 \cdot \left(9 x + 2\right)}{3 \cdot \left(2 x + 1\right)}$$
\begin{center}
\noindent\rule{8cm}{0.4pt}
\end{center}
Вариант $184$


$$f(x) = \frac{2 x + 3}{3 x - 2}$$
\begin{center}
\noindent\rule{8cm}{0.4pt}
\end{center}
Вариант $185$


$$f(x) = \frac{\left(x + 2\right) \operatorname{atan}{\left(\frac{x}{3} \right)} - \frac{2}{3}}{x + 2}$$
\begin{center}
\noindent\rule{8cm}{0.4pt}
\end{center}
Вариант $186$


$$f(x) = \frac{2 x - 3}{2 \cdot \left(3 x - 1\right)}$$
\begin{center}
\noindent\rule{8cm}{0.4pt}
\end{center}
Вариант $187$


$$f(x) = \frac{3 e^{- x}}{2 x - 1}$$
\begin{center}
\noindent\rule{8cm}{0.4pt}
\end{center}
Вариант $188$


$$f(x) = \frac{2 \cdot \left(3 x^{2} - 2 x - 1\right)}{3 \cdot \left(4 x + 1\right)}$$
\begin{center}
\noindent\rule{8cm}{0.4pt}
\end{center}
Вариант $189$


$$f(x) = \frac{\left(x + 3\right) \left(2 x - 1\right) + 2}{2 x - 1}$$

\end{document}
